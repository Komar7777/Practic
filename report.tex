
\documentclass[a4paper,12pt]{article}
\usepackage[utf8]{inputenc}
\usepackage[russian]{babel}
\usepackage{geometry}
\geometry{margin=1in}
\usepackage{booktabs}
\usepackage{amsmath}
\usepackage{amsfonts}
\usepackage{graphicx}
\usepackage{natbib}
\usepackage{hyperref}
\usepackage{noto}

\begin{document}

\title{Отчет по прогнозированию неисправности компьютерных компонентов}
\author{Комаров Даниил Иванович}
\date{\today}
\maketitle

\section{Введение}
Данный отчет представляет результаты анализа данных и производительности моделей машинного обучения для прогнозирования неисправности компьютерных компонентов. Использованы алгоритмы Random Forest, Gradient Boosting, MLP, SVM, KNN и ансамблевая модель стэкинга.

\section{Методология}
\begin{itemize}
    \item Датасет: POLOMKA.csv
    \item Предобработка: кодирование категориальных переменных, нормализация, удаление выбросов
    \item Подбор признаков: алгоритм Boruta
    \item Оценка: Accuracy, Precision, Recall, F1, Log Loss, ROC-AUC
\end{itemize}

\section{Результаты}
\subsection{Сравнение моделей}
\begin{table}
\caption{Сравнение производительности моделей}
\label{tab:model_comparison}
\begin{tabular}{lrrrrrr}
\toprule
Model & Accuracy & Precision & Recall & F1 & Log Loss & ROC-AUC \\
\midrule
Random Forest & 0.979 & 0.852 & 0.377 & 0.523 & 0.076 & 0.933 \\
Gradient Boosting & 0.977 & 0.742 & 0.377 & 0.500 & 0.072 & 0.951 \\
MLP & 0.976 & 0.676 & 0.410 & 0.510 & 0.087 & 0.922 \\
SVM & 0.970 & 0.000 & 0.000 & 0.000 & 0.094 & 0.884 \\
KNN & 0.974 & 0.765 & 0.213 & 0.333 & 0.387 & 0.816 \\
Stacking & 0.979 & 0.788 & 0.426 & 0.553 & 0.082 & 0.946 \\
\bottomrule
\end{tabular}
\end{table}


\subsection{Интерпретация}
Анализ SHAP-значений показал, что наиболее значимыми признаками являются Rotational speed [rpm], Torque [Nm] и Tool wear [min]. Подбор признаков с помощью Boruta подтвердил важность этих признаков, исключив менее значимые, такие как Type.

\section{Выводы}
На основе анализа производительности моделей можно сделать вывод, что Random Forest и ансамблевая модель стэкинга показали наилучшие результаты с точки зрения точности и ROC-AUC. Дальнейшая работа может включать оптимизацию гиперпараметров, тестирование на дополнительных данных и интеграцию с реальными системами мониторинга.

\section{Рекомендации}
\begin{itemize}
    \item Использовать Random Forest или Stacking для реальных приложений.
    \item Проводить регулярное обновление моделей с новыми данными.
    \item Интегрировать систему с датчиками для мониторинга в реальном времени.
\end{itemize}

\bibliographystyle{plain}
\bibliography{references}

\end{document}
